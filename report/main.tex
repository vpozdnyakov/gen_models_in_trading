\documentclass{article}
\usepackage[utf8]{inputenc}
\usepackage[english]{babel}
\usepackage[margin=3cm]{geometry}
\usepackage[nottoc]{tocbibind}
\usepackage{amssymb}
%\usepackage{amsmath}
\usepackage{amsmath}

\usepackage[square,numbers]{natbib}
\bibliographystyle{abbrvnat}

\usepackage{sectsty}
\subsubsectionfont{\normalfont\itshape}

\usepackage{amsthm}
\theoremstyle{definition}
\newtheorem{definition}{Definition}[section]

\DeclareMathOperator{\Var}{Var}

\renewcommand\thesubsubsection{\Alph{subsubsection}}

\title{Application of Generative Models \\ in Commodity Trading}

\author{Vitaliy Pozdnyakov}
\date{}

\begin{document}

\maketitle

\begin{abstract}
    Abstract. Sample references: Ref 1 \cite{renscen}, ref 2 \citet{jebara}.
\end{abstract}

\section{Introduction}
Introduction, related works, business case

\section{Related works}
Related works

\section{Methodology}
Methodology

\subsection{Stochastic differential equations}

\theoremstyle{definition}
\begin{definition}
    A stochastic process is a random variable of the form
    $$X_t(\omega): T \times \Omega \to \mathbb R$$
\end{definition}

\begin{definition}
    A stochastic process $W = W_0, W_1, \dots, W_T$ is called a Wiener process if the following properties hold:
    \begin{enumerate}
        \item $W_0 = 0$ with probability 1
        \item $\mathbb E W_t = 0$
        \item $\Var[W_b - W_a] = b - a$ for all $0\leq a \leq b \leq T$
        \item $W_b - W_a \sim \mathcal N (0, b-a)$
        \item $W_b - W_a$ and $W_d - W_c$ are independent for all $a \leq b \leq c \leq d$
    \end{enumerate}
\end{definition}

\begin{definition}
    A stochastic differential equation is an equation of the form
    $$\mathrm d S_t = \mu S_t \mathrm dt + \sigma S_t \mathrm d W_t$$
    where $\mu$ and $\sigma$ are some constants or some fucntions which represent the interest rate of nonrisky activities and is the market volatility respectively.
\end{definition}

\begin{definition}
    An Ito process is a stochastic process such that
    $$X_t = X_0 + \int_0^t g(s) \mathrm ds + \int_0^t h(s) \mathrm dW_s$$
    where $g(t, \omega)$ and $h(t, \omega)$ are random functions such that
    $$P \left( \int_0^T |g(w, t)| \mathrm dt  < \infty\right) = 1 \text{ and } P \left( \int_0^T h(w, t)^2 \mathrm dt  < \infty\right) = 1$$
\end{definition}

\begin{definition}
    A diffusion process is the solution of SDE of the form
    $$\mathrm d X_t = b(t, X_t) \mathrm  d t + \sigma(t, X_t) \mathrm d W_t$$
    where some deterministic functions $b$ and $\sigma$ are called the \emph{drift} and the \emph{diffusion} coefficient of SDE such that
    $$P\left\{\int_0^T \sup_{|x|\leq \mathbb R}(|b(t, x)| + \sigma^2(t, x))\mathrm dt < \infty\right\} = 1$$
\end{definition}

\subsubsection{Black-Scholes-Merton or geometric Brownian motion}

\begin{definition}
    The  Black-Scholes-Merton is described by the stochastic differential equation of the form
    $$\mathrm d X_t = \theta_1 X_t \mathrm d t + \theta_2 X_t \mathrm d W_t, X_0=x_0$$
    with $\theta_2 > 0$. The parameter $\theta_1$ represents the constant interest rate and $\theta_2$ --- the volatility of risky activities.
\end{definition}

The explicit solution is $$X_t = x_0 \exp\left\{ \left(\theta_1 - \frac{1}{2}\theta_2^2 \right) t + \theta_2 W_t \right\}$$

\subsubsection{Cox-Ingersoll-Ross}
Cox-Ingersoll-Ross

\subsection{Generative and Discriminative models}
Generative and Discriminative models

\subsubsection{Bayesian Inference}
Bayesian Inference

\subsubsection{Support Vector Machines}
Support Vector Machines

\subsubsection{Generative adversarial networks}
Generative adversarial networks

\section{Case study}
Case study

\subsection{Model description}
Model description

\subsection{Model training}
Model training

\subsection{Dataset description}
Dataset description

\subsection{Data preprocessing}
Data preprocessing

\subsection{Quality metrics and results of testing}
Quality metrics and results of testing

\section{Conclusion}
Conclusion

\bibliography{main}

\end{document}
