\documentclass{article}
\usepackage[utf8]{inputenc}
\usepackage[english]{babel}
\usepackage[margin=3cm]{geometry}
\usepackage[nottoc]{tocbibind}
\usepackage{amssymb}
%\usepackage{amsmath}
\usepackage{amsmath}

\usepackage[square,numbers]{natbib}
\bibliographystyle{abbrvnat}

\usepackage{sectsty}
\subsubsectionfont{\normalfont\itshape}

\usepackage{amsthm}
\theoremstyle{definition}
\newtheorem{definition}{Definition}[section]

\DeclareMathOperator{\Var}{Var}

\renewcommand\thesubsubsection{\Alph{subsubsection}}

\title{Application of Generative Models \\ in Commodity Trading}

\author{Vitaliy Pozdnyakov}
\date{}

\begin{document}

\maketitle

\begin{abstract}
    Abstract. Sample references: Ref 1 \cite{renscen}, ref 2 \citet{jebara}.
\end{abstract}

\section{Introduction}
Introduction, related works, business case

\section{Methodology}

\subsection{Stochastic differential equations}

\theoremstyle{definition}
\begin{definition}
    A stochastic process is a random variable of the form
    $$X(t, \omega): T \times \Omega \to \mathbb R$$
\end{definition}

\begin{definition}
    A stochastic process $W = \{W(t)\}_{t\geq0}$ is called a Wiener process if the following properties hold:
    \begin{enumerate}
        \item $W(0) = 0$ with probability 1
        \item $\mathbb E W(t) = 0$
        \item $\Var[W(b) - W(a)] = b-a$ for all $0\leq a \leq b \leq T$
        \item $W(b) - W(a) \sim \mathcal N (0, b-a)$
        \item $W(b) - W(a)$ and $W(d) - W(c)$ are independent for all $a \leq b \leq c \leq d$
    \end{enumerate}
\end{definition}

\begin{definition}
    A stochastic differential equation is an equation of the form
    \begin{equation}
        \mathrm d S(t) = \mu S(t) \mathrm dt + \sigma S(t) \mathrm d W(t)
    \end{equation}
\end{definition}

\subsubsection{Black-Scholes-Merton or geometric Brownian motion}
Black-Scholes-Merton or geometric Brownian motion

\subsubsection{Cox-Ingersoll-Ross}
Cox-Ingersoll-Ross

\subsection{Generative and Discriminative models}
Generative and Discriminative models

\subsubsection{Bayesian Inference}
Bayesian Inference

\subsubsection{Support Vector Machines}
Support Vector Machines

\subsubsection{Generative adversarial networks}
Generative adversarial networks

\section{Case study}

\subsection{Dataset description}
Dataset description

\subsection{Data preprocessing}
Data preprocessing

\subsection{Model description}
Model description

\subsection{Model training}
Model training

\subsection{Quality metrics and results of testing}
Quality metrics and results of testing

\section{Conclusion}
Conclusion

\bibliography{main}

\end{document}
